\chapter*{論文概要}
\thispagestyle{empty}
\begin{flushleft}
  \textbf{提出日}   2019年2月20日\\
  \textbf{専攻}  物理・数理学科\\
  \textbf{指導教員}  北野 晴久教授 \\
  \textbf{学籍番号}  15115062\\
  \textbf{氏名}  小松優基\\

\end{flushleft}
\textbf{論文題目} \textbf{空洞量子電磁力学実験のためのマイクロ波空洞共振器の設計}\\
\thispagestyle{empty}

\textbf{論文要旨}

\begin{abstract}
量子コンピューターに使われる量子ビットの有力候補の一つに「超伝導回路」によるものがある。超伝導回路ではジョセフソン接合を用いて量子もつれ状態を実現しているが、情報の保存時間であるコヒーレンス時間が現在数十マイクロ秒程度であり、0.1K以下の極低温下でしか動作しないなどの課題も多い。
 
 北野研究室では、2015年度の結果から、固有ジョセフソン接合(IJJ)素子の高次スイッチ現象において、41.5GHzのマイクロ波照射下でスイッチング電流分布の二重ピーク構造が観測され、離散化したエネルギー準位の形成が示唆された。前述の課題解決に向けて高温超伝導体のIJJを用いた超伝導量子ビットの実現を目指し、本研究では、40~GHzにおける、空洞量子電磁力学実験(cavity QED)を行うためのマイクロ波空洞共振器の設計を電磁界解析シュミレータ(MW-Studio,CST社)を用いて行った。本研究では、同軸ケーブルとの結合を考慮し37~51GHzで共振周波数を変化させる機能を持った共振器の設計ができた。今後の課題として、より高いQ値になるように、電磁場の結合方法を探していくことが必要となる。また、シミュレーションのみでは微小な試料を入れた際の応答までは計算できないため、実際に今回試したモデルを元に空洞共振器を試作し、共振器自体の特性を調査する必要がある。
\end{abstract}
