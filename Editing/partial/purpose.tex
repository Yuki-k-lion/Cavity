\chapter{研究目的}
前述の通り、IJJ素子の特性を調べるため、40GHz以上で共振周波数の調整が可能な空洞共振器の設計が目的である。
\section{昨年度の研究結果}
昨年度は、空洞共振器の大きさと誘電体を挿入した際の共振周波数を調べていた。 [引用つける]

\section{昨年度の課題}
誘電体を挿入した際に共振周波数が下がることが観測されたが、以下の理由により実用的なモデルとは言えなかった。
\subsection{共振周波数の調整方法}
具体的にどのようにして共振周波数を調整するのかが不明瞭。

誘電体を挿入することで共振周波数を変化させることができることはわかったが、昨年度使用していた誘電体の誘電率は100程度であり、マイクロメータを使用して共振周波数を調整したとしても、細く共振周波数を変化させられる仕様ではなかった。
\subsection{外界との結合方法}
空洞共振器は単体で共振現象を起こすものではなく、必ず外部からマイクロ波を入射する必要がある。外界と共振器の電磁場を結合させるために、同軸ケーブルを使用する。昨年度作成していたモデルでは、一辺の長さが1mm程度であり、同軸ケーブルを接続することができないもモデルであった。
\subsection{試料の設置位置が不明瞭}
誘電体挿入時内部の電磁場がどのような分布になるのかが明らかになっていなかったため、試料の設置位置も不明瞭であった。


\section{本研究の目的}
昨年度の研究結果を受けて、改めて本年度の研究目的を整理すると、

\begin{itemize}
  \item 誘電体による共振周波数の変化の傾向を明らかにする。
  \item 同軸ケーブルを接続したモデルを作成する。
  \item 誘電体挿入時の電場変化を考慮して試料設置位置を決める。
\end{itemize}

以上3点を満たした、 共振周波数40GHz以上で調整可能な、実用的な空洞共振器の設計を行うことが本研究の目的である。
